\documentclass[UTF8]{ctexart}
\author{赵乙麒}
\title{基于深度学习的股价预测解决方法分析研究}
\begin{document}
\maketitle
\section{绪论}
\subsection{选题背景与意义}
股票(stock)是股份公司发行的所有权凭证,是股份公司为筹集资金而发行给各个股东作为持股凭证并借以取得股息和红利的一种有价证券。每股股票都代表股东对企业拥有一个基本单位的所有权。每家上市公司都会发行股票。股票市场是已经发行的股票转让、买卖和流通的场所。17世纪荷兰和英国成立了海外贸易公司。这些公司通过募集股份资本而建立。在经历了4个多世纪的今天,股票市场已经进入了大多数国家。而且在当今世界经济格局中,各个国家的股市已经拥有了不可或缺、举足轻重的地位。对于在股市中投资的人来讲,赚钱是他们的首要目的。但是股市有着高风险性,一句“股市有风险,入市需谨慎”劝退了很多想进入股市分一杯羹的人。对于投资公司来讲,若他们能预知股市未来的走向,毋庸置疑,他们就可以获得利润。所以,股票价格的预测就成了上百年来人们追求的目标。在深度学习理论成熟之前,人们在股市预测领域主要采取一些传统统计学、微波转换\cite{Ramsey1999}、事件分析\cite{Verma2017}等方法预测股价。但由于影响股市的因素过多(政策、经济发展情况、新闻等),使这些传统方法有局限性。1956年,几个计算机科学家相聚在达特茅斯会议(Dartmouth Conferences),提出了“人工智能”的概念。其后,人工智能就一直萦绕于人们的脑海之中,并在科研实验室中慢慢孵化。之后的几十年,人工智能一直在两极反转,或被称作人类文明耀眼未来的预言;或者被当成技术疯子的狂想扔到垃圾堆里。坦白说,直到2012年之前,这两种声音还在同时存在。过去几年,尤其是2015年以来,人工智能开始大爆发,有了突破性的进展\cite{LeCun2015,Schmidhuber2015}。很大一部分是由于GPU的广泛应用,使得并行计算变得更快、更便宜、更有效。当然,无限拓展的存储能力和骤然爆发的数据洪流(大数据)的组合拳,也使得图像数据、文本数据、交易数据、映射数据全面海量爆发。之后计算机科学家们提出了机器学习、深度学习等想法,进而很多研究者投入这方面的研究,很多深度学习算法被提出,使得股票市场的研究燃起了新的火焰。股票数据和其他的类似于图片、文本等的数据不一样,它是一种时间序列数据,前面的数据会影响到后面的数据。针对这种时间序列数据,循环神经网络(RNN)、长短期记忆网络(LSTM)等神经网络结构应运而生。股票市场研究领域也因为这些网络结构的兴起而有着强大的生命力。虽然深度学习在股票市场预测的研究中相比一些传统方法有优势,但深度学习算法未被应用于更广泛的股市预测领域。如今的股票市场研究领域,大多在研究、预测标准普尔指数和纳斯达克指数。这些新提出的深度学习算法是否能同样使用于中国股市未可知。所以,本文以这作为落脚点和出发点,深入探讨如今越来越先进的深度学习算法,是否能很好地预测中国股市未来的发展。
\subsection{国内外研究现状}
近年来,金融市场在我国发挥着的作用越来越显著,随着国民经济的发展和金融服务业的完善,在金融市场中起着关键总用的股票市场已经引起了国内外学者和投资者的关注。他们定期提出各种可应用于实践的理论,试图预测市场趋势\cite{Lahmiri2015,Chiang2015,Seddon2017,Zhou2016,Ichinose2018}。在如今深度学习发展的基础上\cite{Gers2002,Hinton2006,Jiang2018,Kim2015,Kuremoto2014,Torres2017},神经网络在模式识别、金融证券等领域得到了广泛的应用。最早还要追溯到1988年,White和Helbert首次将 BP 神经网络模型应用于股票市场序列的处理和预测中,其使 IBM 公司股票日收益率作为实证研究的对象,最终得出预测结果十分理想\cite{White1988}。之后Bernardete Ribeiro、Noel Lopes对限制玻尔兹曼机(RBM)、支持向量机(SVM)和深度信念网络(DBN)三种模型对公司财务状况进行分析,结果表明(DBN)模型可以在描述财务状况表征更好的特性\cite{Ribeiro2011}。现已有多篇论文使用LSTM、RNN等神经网络算法研究股指、股价等相关信息\cite{Pang2018,Chong2017,Bao2017,Chen2015,Fischer2018,Hsieh2011,Huynh2017,Liu2017},这些算法显示除了在股票市场时间序列预测中的优势。例如,在早期的工作中\cite{Kamijo1990}已经使用RNN代替了波动性预测模型来预测股价。然而,如今现有的股票市场分析领域,对数据的分析并不完备。
\subsection{本文主要内容}

\subsection{本文的组织结构与技术路线}
①分析股票市场的不同数据的特点,影响因素等。
②针对股票市场的不同数据建模。
③实现不同算法(神经网络)的代码编写。
④找到相关数据,形成训练集。
⑤使用不同算法对这些数据进行预测。
\section{深度学习理论基础}
\subsection{传统神经网络}
\subsection{深度神经网络}
\subsection{卷积神经网络CNN}
\subsection{循环神经网络RNN}
\subsection{长短期记忆网络LSTM}
\subsection{神经网络训练的优化方法}
\section{模型构建}
\subsection{Python}
\subsection{PyTorch介绍}
\subsection{基于RNN的模型构建}
\subsection{基于LSTM的模型构建}
\subsection{输入特征}
\section{股市数据的选取}
\subsection{数据来源}
\subsection{IT领域企业股价}
\subsection{样本选取}
\section{实验分析}
\subsection{优化方法}
\subsection{参数设置}
\subsection{整体效果}
\subsection{指标分析}
\subsection{几种解决方法对比}
\section{结论}
\bibliographystyle{plain}
\bibliography{../bib/trade.bib}
\end{document}