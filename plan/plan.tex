\documentclass[UTF8]{ctexart}
\author{赵乙麒}
\title{基于深度学习的股价预测解决方法分析研究}
\begin{document}
\maketitle
\section{绪论}
\subsection{选题背景与意义}
最早的股市公司是17世纪荷兰和英国成立的海外贸易公司。在经历了4个多世纪的今天,股票市场已经进入了大多数国家。而且在当今世界经济格局中,各个国家的股市已经拥有了不可或缺、举足轻重的地位。对于在股市中投资的人来讲,赚钱是他们的首要目的。但是股市有着高风险性,一句“股市有风险,入市需谨慎”劝退了很多想进入股市分一杯羹的人。对于投资公司来讲,若他们能掌握股市未来的走向,那他们就可以获得利润。所以,股票价格的预测就成了上百年来人们追求的目标。在深度学习理论成熟之前,人们在股市预测领域主要采取一些微波转换\cite{Ramsey1999}、事件分析\cite{Verma2017}等方法预测股价。但由于影响股市的因素过多(政策、经济发展情况、新闻等),使这些传统方法有局限性。在近几年来,深度学习技术有了突破性的进展\cite{LeCun2015,Schmidhuber2015},很多深度学习算法被提出,使的股票市场的研究燃起了新的火焰。虽然深度学习在股票市场预测的研究中相比一些传统方法有优势,但深度学习算法未被应用于更广泛的股市预测领域。如今的股票市场研究领域,大多在研究、预测标准普尔指数和纳斯达克指数。这些新提出的深度学习算法是否能同样使用于中国股市未可知。所以,本文以这作为落脚点和出发点,深入探讨如今越来越先进的深度学习算法,是否能很好地预测中国股市未来的发展。
\subsection{国内外研究现状}
近年来,金融市场在我国发挥着的作用越来越显著,随着国民经济的发展和金融服务业的完善,金融市场已经引起了国内外学者和投资者的关注。他们定期提出各种可应用于实践的理论,试图预测市场趋势\cite{Lahmiri2015,Chiang2015,Seddon2017,Zhou2016,Ichinose2018}。在如今深度学习发展的基础上\cite{Gers2002,Hinton2006,Jiang2018,Kim2015,Kuremoto2014,Torres2017},神经网络在模式识别、金融证券等领域得到了广泛的应用。此外,它也显示出了在股票市场预测中的优势。现已有多篇论文使用LSTM、RNN等神经网络算法研究股指、股价等相关信息\cite{Pang2018,Chong2017,Bao2017,Chen2015,Fischer2018,Hsieh2011,Huynh2017,Liu2017}。例如,在早期的工作中\cite{Kamijo1990}已经使用RNN代替了波动性预测模型来预测股价。然而,如今现有的股票市场分析领域,对数据的分析并不完备。
\subsection{本文主要内容}
\subsection{本文的组织结构与技术路线}
①分析股票市场的不同数据的特点,影响因素等。
②针对股票市场的不同数据建模。
③实现不同算法(神经网络)的代码编写。
④找到相关数据,形成训练集。
⑤使用不同算法对这些数据进行预测。
\section{深度学习理论基础}
\subsection{传统神经网络}
\subsection{深度神经网络}
\subsection{卷积神经网络CNN}
\subsection{循环神经网络RNN}
\subsection{长短期记忆网络LSTM}
\subsection{神经网络训练的优化方法}
\section{模型构建}
\subsection{Python}
\subsection{PyTorch介绍}
\subsection{基于RNN的模型构建}
\subsection{基于LSTM的模型构建}
\subsection{输入特征}
\section{股市数据的选取}
\subsection{数据来源}
\subsection{IT领域企业股价}
\subsection{样本选取}
\section{实验分析}
\subsection{优化方法}
\subsection{参数设置}
\subsection{整体效果}
\subsection{指标分析}
\subsection{几种解决方法对比}
\section{结论}
\bibliographystyle{plain}
\bibliography{../bib/trade.bib}
\end{document}